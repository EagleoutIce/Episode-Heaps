\errorcontextlines 999999
\usepackage[ngerman]{babel}

\usepackage[%
    sopra-listings={encoding,cpalette,highlights},%
    sopra-tables={cpalette}, color-palettes={addons},%
    lecture-bibliography={biber,style=numeric-comp},%
    util, lithie-boxes={germanenv,koma,overwrite},%
    lithie-task-boxes={cpalette}, lecture-links={patchurl},%
    lecture-personal-resize,
    lecture-registers={disable}% would interfere with beamer
]{lithie-util}

\usepackage{forest}

\makeatletter
\sol@list@define@styles{%
  {keywordA: \@declaredcolor{sol@colors@lst@keywordA}\bfseries},%
  {keywordC: \@declaredcolor{sol@colors@lst@keywordB}\bfseries},%
}
\makeatother

\RestyleAlgo{plain}
\lstset{lineskip=5.5pt}
\lstfs{10}

\DefinePalette{Heaps}
{Grün,grünlich: RGB(37,126,99)}% gen viridian
{Magenta,magenta: RGB(138, 49, 104)}
{Gelb,gelblich: RGB(255, 200, 87)}
{Blau,bläulich: RGB(21, 92, 148)}
\SetShadeContrast{45}
\UsePalette{Heaps}

\usetheme[libs,nofootfade,centerfoot]{dividing-lines}
\SetColorProfile*{paletteA}{paletteB}{paletteC}

\usetikzlibrary{arrows.meta,decorations,decorations.pathreplacing,shapes.multipart,tikzmark,shapes.symbols}

\def\info#1{\bgroup\scriptsize\textcolor{gray}{(#1)}\egroup}
\SetAllLinkStyle{#1}
\def\fillfont{\def\mdseries@sf{medium}\sffamily}
\colorlet{lgray}{lightgray!48!white}
\tikzset{
    ldesc/.style={gray,font=\sffamily\sbfamily},
    lrel/.style={fill=white,rounded corners,minimum width=28mm,minimum height=7.5mm,align=center},
    lrel2/.style={fill=white,rounded corners,minimum width=28mm,minimum height=7.5mm*2,align=center},
    lsf/.style={fill=white,rounded corners,minimum width=28mm,minimum height=7.5mm*2,align=center,
        rectangle split, rectangle split parts=2},
    blob/.style={circle,draw, minimum size=1.9em,align=center},
    lblob/.style={blob,writ,font=\fillfont,#1},%
    lblob/.default={fill=lgray, draw=lgray,text=black},
    every picture/.append style={line join=round,line cap=round},%
    every node/.append style={font=\sffamily},%
    lblock/.style={block,writ,font=\fillfont,#1},%
    lblock/.default={fill=lgray, draw=lgray,text=black},
    block/.style={rectangle,draw, align=center, minimum height=1.25em,rounded corners=1.55pt},%
}

\newcommand\parallelcontent[3][t]{%
    \begin{columns}[#1]
    \begin{column}{.475\linewidth}#2\end{column}\hfill
    \begin{column}{.475\linewidth}#3\end{column}
    \end{columns}
}

\usepackage[glows]{tikzpingus}
\usetikzlibrary{decorations.text,graphs}
\hypersetup{colorlinks=false}

\title{Fantastische Heaps}
\subtitle{\ldots\ und wo sie zu finden sind.\\Datastructure Love!}
\institute{SP, Universität Ulm}

\author{Florian Sihler}
\email{florian.sihler@uni-ulm.de}

\date{\today}
\outro{Ulm, \today}
\license[]{All rights reserved}

\def\PreTitlepage{\begingroup%
\let\oldinserttitle\inserttitle% allow it to be white on second slide
\let\oldinsertsubtitle\insertsubtitle% allow it to be white on second slide
\colorlet{PINGU@WHITE}{pingu@white}% hacksies for the whites
\only<2|handout:2>{\def\inserttitle{\color{pingu@white}\oldinserttitle}\def\insertsubtitle{\textcolor{pingu@white}{\oldinsertsubtitle}}}%
\onslide<2|handout:2>{%
\savebox0{\tikz{\pingu[lightsaber left=paletteC!90!white,left item angle=-20,eyes=shiny,right wing grab,tie=paletteA,tie dots,body=paletteA!20!black,lightsaber left outer glow factor=.09,headband]}}%
\begin{tikzpicture}[overlay,remember picture]%
    % beamer does not support changes of full background easily. so we do hacksies
    \pgfonlayer{background}
    \path[fill,black!99] (current page.north west) rectangle (current page.south east);
    \endpgfonlayer
    \node[above right=.15cm,scale=.65,xshift=.15cm] (pingu) at(current page.south west) {\usebox0};
    \node[below right,pingu@white,text width=.6\paperwidth,align=flush left] at(pingu.north east){Officially supported by the Pingu-Foundation for Emotional Support. There will be light and there will be cute waddlers!};
\end{tikzpicture}}%
}
\def\PostTitlepage{\endgroup}

\addbibresource{./references.bib}


\setcounter{tocdepth}{4}
\newsavebox\pinguA \newsavebox\pinguB \newsavebox\pinguC

\usepackage[link]{qrcode}
\outroright{\smash{\raisebox{1.33cm/2}{\qrcode[height=1.33cm]{https://github.com/EagleoutIce/Episode-Heaps}}}\begin{tikzpicture}[remember picture,overlay]
    \node[above left,btdl@color@white,scale=.475] at (current page.south east) {\href{https://github.com/EagleoutIce/Episode-Heaps}{Slides and \LaTeX-sources on GitHub!}};
\end{tikzpicture}}

\newcommand*\heap[2][]{\downsize{\linewidth}{\begin{forest}for tree={lblob,minimum width=2.5em,s sep=4em-level*.5em,edge={line width=3pt,lgray},#1}#2\end{forest}}}

\makeatletter
% 4 empty => ship arbitrary with 5
\long\def\hsStep{\@ifstar\@hsStep\@@hsStep}
\long\def\@hsStep#1#2#3#4#5{\@@@hsStep{#1}{#2}{#3}{#4}{#5}{Kite-}}
\long\def\@@hsStep#1#2#3#4#5{\@@@hsStep{#1}{#2}{#3}{#4}{#5}{Kite-Kite}}
\long\def\@@@hsStep#1#2#3#4#5#6{%
\begin{center}
\savebox0{\heap{#1}}\parbox[c][12em]{16.25em}{\centering\scalebox{.9}{\usebox0}\vfill\null}\hfill\begin{tikzpicture}
\foreach[count=\i,remember=\i as \li (initially 0)] \l in {#2} {
    \xdef\curmax{\i}
    \ifnum\li=0
        \node[lblock,minimum size=2.25em,fill=white,very thick,right,outer sep=0pt] (1) at (0,0) {\l};
    \else
        \node[lblock,minimum size=2.25em,fill=white,right,outer sep=0pt,very thick] (\i) at (\li.east) {\l};
    \fi
}
\draw[lgray,rounded corners=1.55pt,ultra thick] (1.south west) rectangle (\curmax.north east);
\pgfinterruptboundingbox
\foreach \i in {1,...,\curmax} {
    \ifnum\i=#3
    \node[below=1mm,paletteA,font=\sbfamily,scale=.6] (d\i) at (\i.south) {\strut\the\numexpr\i-1};
    \else
    \node[below=1mm,gray,font=\sbfamily,scale=.6] (d\i) at (\i.south) {\strut\the\numexpr\i-1};
    \fi
}
\ifx!#4! #5\else
    \draw[gray,thick,#6] (d#4.south) to[bend right=20] (d#5.south);
\fi
\endpgfinterruptboundingbox
\end{tikzpicture}
\end{center}}
\makeatother

\hfuzz=5cm